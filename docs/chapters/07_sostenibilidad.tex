\chapter{Sostenibilidad}\label{sostenibilidad}
\section{Autoavaluación}

Tras realizar la encuesta sobre la sostenibilidad, me he dado cuenta de que poseo unas nociones muy básicas sobre esta, especialmente cuando se trata de mi ámbito profesional. A pesar de tener una mínima base teórica, a lo largo de mi carrera universitaria apenas he tenido la oportunidad de aplicarla en un proyecto cuya magnitud sea crítica.

Si bien es cierto que en ocasiones he pensado en el impacto social que tienen o podrían tener ciertos productos relacionados con la informática, es distinto cuando se trata de la dimensión económica y ambiental, las cuales no se han tenido tanto en cuenta. Esto se debe en gran parte a que los proyectos en los que he participado no van más allá de ser aplicaciones sencillas, las cuales requieren poca inversión en recursos y no trascienden del ámbito académico, por lo que el impacto económico y ambiental es mínimo.

La realización de un trabajo de fin de grado es una buena oportunidad para cambiar esta situación, ya que permite poner estas nociones en práctica durante la realización del proyecto, incitando a abordar una solución sostenible tanto en la dimensión económica como la ambiental y social.

\section{Dimensión económica}

Sobre el coste estimado para la realización del proyecto, teniendo en cuenta la finalidad que tiene en los ámbitos de entretenimiento e investigación, además de ser una manera bastante innovadora de abordar el problema planteado, pienso que es un coste justificado en relación a los beneficios que aporta.

Generalmente, desarrollar un simulador de pádel en el que los jugadores puedan adquirir comportamientos complejos suele ser una tarea bastante laboriosa, sobre todo cuando no se emplean las técnicas adecuadas, lo que supone un aumento en el coste de personal. La solución planteada en este proyecto, por lo contrario, busca reducir este coste simplificando el proceso de desarrollo.

Si bien es cierto que el entrenamiento de un modelo resulta costoso, una vez finalizado e integrado en el producto final, transferir el modelo y utilizarlo resulta ser un proceso muy eficiente. Una contraparte es, sin embargo, su coste de mantenimiento. Esto se debe específicamente a que se trata de una aplicación que debe mantenerse al tanto de la normativa del pádel, aunque sea poco probable que esta cambie. Además, hay que tener en cuenta que las estrategias de los jugadores de pádel también van evolucionando a lo largo del tiempo, por lo que podría surgir la necesidad de reentrenar agentes con datos actualizados.

\newpage

\section{Dimensión ambiental}

El desarrollo de este proyecto se ha llevado a cabo principalmente con un ordenador de sobremesa, y se ha utilizado ocasionalmente un portátil cuando ha sido conveniente. Por lo tanto, el impacto ambiental se ha reducido a la fabricación del equipo de trabajo y su consumo.

Respecto al espacio de trabajo, se ha trabajado principalmente en un hogar dotado de paneles solares, por lo que parte del consumo eléctrico ha provenido de energías renovables. Puntualmente, se ha optado por trabajar en bibliotecas públicas a causa de algunos inconvenientes en el hogar.

Para los desplazamientos necesarios para las reuniones, las cuales han sido mayoritariamente remotas, se ha utilizado exclusivamente el transporte público para tratar de reducir la huella de carbono. Además, se ha trabajado únicamente en digital, por lo que el uso de recursos naturales ha sido mínimo.

En cuanto al impacto ambiental del producto desarrollado, aunque el entrenamiento de agentes haya requerido un consumo ligeramente más elevado, la ejecución supone un consumo energético parecido al de un programa cualquiera, y la transferencia de los modelos entrenados no supone ningún impacto negativo.

\section{Dimensión social}

A nivel personal, como alguien que ha crecido alrededor de los videojuegos y dado mi interés por el aprendizaje automático, rama en la que me gustaría especializarme, este proyecto ha sido el escenario perfecto para, por una parte, tener una idea general sobre cómo funcionan los videojuegos internamente y, por otra, profundizar mis conocimientos en el ámbito del aprendizaje por refuerzo.

En cuanto a la vida útil del producto, actualmente existen muy pocos simuladores o videojuegos de pádel, en los que generalmente no se utilizan técnicas de aprendizaje por refuerzo para atribuir inteligencia artificial a los personajes no jugadores. Es por ello que el producto desarrollado permite darle otro enfoque al uso de estas técnicas, ofreciendo una alternativa que logre comportamientos más complejos en un menor tiempo de desarrollo.

Si bien es cierto que este proyecto no es una necesidad prioritaria en la sociedad, para aquellas personas interesadas en el mundo del pádel se podría considerar como otra forma de entretenimiento y/o aprendizaje. Después de todo, un simulador permite facilitar la enseñanza aportando un apoyo visual, en el que se podría mostrar situaciones que no se suelen dar en el pádel y se podría hacer un análisis de estas. Además, para otros investigadores interesados en el aprendizaje por refuerzo, este proyecto puede servir como referencia para entender cómo funciona y experimentar con ello.