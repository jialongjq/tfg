\chapter{Planificación final}

\section{Cambios en la planificación}

A continuación se comentan los cambios que se han producido en la planificación, tras haber finalizado el desarrollo del proyecto.

El bloque T2 se había planificado con una duración breve, ya que no pretendía ser un estudio demasiado extenso sobre el estado del arte. Sin embargo, con el estudio de los algoritmos que se pretendían utilizar, surgió la necesidad de profundizar mucho más en la teoría del aprendizaje por refuerzo, por lo que la duración se acabó extendiendo bastante.

Los bloques T4 y T5 se habían planificado anticipando que quizás requerirían una implementación específica para los algoritmos que se querían utilizar. En realidad, el uso de los algoritmos de RL e IL en el entorno de ML-Agents es realmente sencillo, ya que se pueden aplicar en el mismo entorno virtual. Salvo la parte del procesamiento de datos del bloque T5, el resto ha consistido finalmente en hacer distintos experimentos en el entorno virtual. Dados estos cambios, lo ideal sería aumentar las horas del bloque T3, ya que engloba toda la implementación del entorno virtual, y reducir la duración de los bloques T4 y T5.

En el bloque T6, la idea era crear una aplicación con una interfaz gráfica propia que permitiera al usuario final hacer pruebas en el simulador. Sin embargo, esto no era viable ya que para hacer cambios específicos en el entorno virtual es necesario abrir el proyecto desde Unity. Como alternativa, se acabó implementando un modo de depuración, para que el usuario pueda comprobar que todo funcione correctamente durante la ejecución del simulador.

\section{Planificación final}

Tras los cambios producidos en la planificación, se ha actualizado el resumen de tareas y el diagrama de Gantt, los cuales se pueden consultar en la Tabla \ref{tab:planificacion-final} y el Apéndice \ref{appendix:b}, respectivamente.

Los bloques T4, T5 y T6 se han renombrado para que sigan un orden cronológico, ya que el desarrollo de la interfaz de depuración se ha realizado antes de la experimentación. Por lo tanto, el bloque T4 consiste ahora en el desarrollo de la interfaz de depuración, mientras que los bloques T5 y T6 en la experimentación con DRL e IL, respectivamente.

\begin{table}[H]
    \centering
    \begin{tabular}{|>{\rowmac}c|>{\rowmac}c|>{\rowmac}c|>{\rowmac}c<{\clearrow}|}
        \hline
        \setrow{\bfseries} ID & Tarea & Tiempo & Dependencias \\ \hline\hline
        \setrow{\bfseries} T1 & Gestión del proyecto & 100h & - \\
        \hline
        T1.1 & Alcance & 30h & [] \\
        T1.2 & Planificación & 15h & [T1.1] \\
        T1.3 & Presupuesto & 7.5h & [T1.2] \\
        T1.4 & Informe de sostenibilidad & 7.5h & [T1.2] \\
        T1.5 & Reuniones & 25h & [] \\
        T1.6 & Documentación & - & [] \\
        T1.7 & Presentación & 15h & [T1.6] \\
        \hline
        \setrow{\bfseries} T2 & Trabajo previo & 100h & - \\
        \hline
        T2.1 & Estudio del estado del arte & 80h & [] \\
        T2.2 & Familiarización con ML-Agents & 20h & [T2.1] \\
        \hline
        \setrow{\bfseries} T3 & Entorno virtual en Unity & 200h & - \\
        \hline
        T3.1 & Diseño del entorno & 50h & [T2] \\
        T3.2 & Implementación & 100h & [T3.1] \\
        T3.3 & Testing & 50h & [T3.2] \\
        \hline
        \setrow{\bfseries} T4 & Interfaz de depuración & 50h & - \\
        \hline
        T4.1 & Diseño de la interfaz & 10h & [T3.1] \\
        T4.2 & Implementación & 30h & [T4.1] \\
        T4.3 & Testing & 10h & [T4.2] \\
        \hline
        \setrow{\bfseries} T5 & Aprendizaje por refuerzo profundo & 30h & - \\
        \hline
        T5.1 & Experimentación & 30h & [T3] \\
        \hline
        \setrow{\bfseries} T6 & Aprendizaje por imitación & 60h & - \\
        \hline
        T6.1 & Procesamiento de datos & 30h & [T3] \\
        T6.2 & Experimentación & 30h & [T6.1] \\
        \hline
        \setrow{\bfseries} - & Total & 540h & - \\
    \hline
    \end{tabular}
    \caption[Resumen de la planificación final]{Resumen de la planificación final. (Elaboración propia)}
    \label{tab:planificacion-final}
\end{table}